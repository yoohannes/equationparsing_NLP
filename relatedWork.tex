Understanding semantics of a text has been a huge research area in natural language processing. Nowadays many of research and industry work on common applications such as customer services or chatbots can be found. From understanding the context to turning instructions into actions, NLP is essential state-of-the-art technology.

Our focus is semantic analysis for information extraction and in that realm, a few works such as  \parencite{kushman2014learning}, \parencite{chambers2011template} and \parencite{chen2020mapping} have been done. They all dealed with finding a meaningful representation of information from an implicit natural language.

Although \parencite{kushman2014learning} work is solving algebraic word problems than formulating the equations, that is one part of it and it incorporates grounding equations using pattern templates. They learn mapping from a wide range of templates, which are induced from the training set where each example in the set have slots to be filled with values from the sentences and unknowns filled with nouns. This method applies to only algebraic word problems with the four arithmetic operators $(x,-,+,/)$ and relies on the word problem structure similarity and requires to draft a number of templates to handle different word problem types. 

\parencite{koncel2015parsing}  also introduces a different approach for the algebraic word problem equation grounding. ALGES, which is given an unlabeled word problem with its solution, maps word problems into equation trees by matching a text to arithmetic operators, and combining noun phrases to operand nodes.

\parencite{roy2016equation} presents a mapping mechanism where a vocabulary of trigger words for variables, quantities are mapped to one or two variables, and are combined with operands to form a binary tree. For each word problem, an algorithm learns the possible trigger output which will be matched to a lexicon of equation representation. However, this also has a limitation of working with single sentences. 

While most of the mentioned works is mostly rule based approaches, \parencite{huang-etal-2018-neural} used seq2seq model to formulate to solve equations with a recursive neural network. In seq2seq models recognized numbers get mapped to a list of number tokens, which afterwards is solved.

\parencite{shi2015automatically}, introduces dolphine language (DOL), as a semantic representation of the text. Further a semantic parser is used to transfer the DOL tree into mathematical expression. From the tree representation a reasoning module is used to map it into math expression.  

Perhaps a more closely related work \parencite{hosseini2014learning}, introduces an algorithm ARIS that will ground the a word problem and solves the equation. Much like the previous papers it deals with arithmetic word problems. Furthermore it breaks down a word problem into fragments of phrases, verbs, entities and containers which will be the basis for solving the problem. While ARIS uses a classifier to label the different category of verbs which is used to emphasize the impact of a chunk of phrase or word extract, the more information extraction is done by a Stanford’s dependency parser, named entity recognizer which is very similar to our approach. ARIS deals with multi sentences word problems but still limited to arithmetic formulations.\\
\par

\begin{tcolorbox}
The initial temperature of a rod $0<x<l$ is an arbitrary function of $x$. The temperatures of the ends are constant:
\begin{equation*}
u(0,t) = U_{1} = \mathrm{const}, \quad u(l,t) = U_{2} = \mathrm{const}, \quad 0<t<\infty.
\end{equation*}
A heat exchange obeying Newton's law takes place at the surface with a medium whose temperature equals $u_{0}=\mathrm{const}$. Find the temperature of the rod. Consider, in particular, the case where $U_{1}=U_{2}=0$.
\end{tcolorbox}\\
As depicted in the box our approach deals majorly with a different domain and complex word problems that combine multiple variables and mathematical representations, and having a of mix pattern based mapping to deep learning could suffice to our problem. However, NLP is one of the most important technology to solve our problem, therefore a more detailed look is needed.
