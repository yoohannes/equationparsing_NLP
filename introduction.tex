\epigraph{Success in creating AI would be the biggest event in human history.}{\textit{Stephen Hawking}}

In the wake of more powerful computers, lower memory costs and broad investments by tech giants, artificial intelligence has been growing in popularity for some time.  In the Deep Learning subcategory, major advances have been made recently, which include facial and speech recognition as well as autonomous driving. Furthermore, in 2016, the computer progam AlphaGo succeeded in beating the world's best Go player Lee Sedol for the first time.\\

Nevertheless, there are areas in which the success of neural networks is limited. One of these topics is the processing of mathematical problems in the fields of engineering or physics. Only a few papers tried to tackle this problem \parencite{wang2017deep} and Wong considered solving linear algebraic one-variable tasks. Wang et al. showed that sequence to sequence (seq2seq) outperforms state-of-the-art statistical approaches. The outcome of these reports make one optimistic that neural networks are a very promising method of choice for solving these tasks. 

For higher mathematical problems no papers cannot be found. Lample and Charton from Facebook considered this typ of problem, but focused on solving mathematical equations as an direct input by help of deep learning. \\

In this report, we consider mathematical expressions appearing in engineering problem formulations. With the help of Natural Language Processing (NLP) the given problem in text form shall be translated to resulting equations.