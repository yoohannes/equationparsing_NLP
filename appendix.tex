In this chapter the focus is on the math problems we consider to solve. The examples of mathematical expression appearing in engineering problem formulations. In general, these expressions can be sub-divided into the following categories:
\begin{enumerate}
\item {\bfseries Variables}: $x$, $t$, $z$, $\epsilon$;
\item {\bfseries Constants}: $v_{0}$, $C_{0}$, $u_{1}=const$, $u_{0}=const$, $U_{0}=const$, $A=const$, $Q$, $x_{0}$, $h$,  $v_{0}=const$, $E_{0}$, $R_{0}$, $b$;
\item {\bfseries Functional expressions}: $\varphi(t)$, $f(x)$, $u(x,t)$, $F(x,t)$;
\item {\bfseries Equations}: $t=0$, $x=l^{''}$, $f(x)=U_{0}=const$, $x=l/2$, $u(0,t)=U_{1}=const$, $u(l,t)=U_{2}=const$, $U_{1}=U_{2}=0$, $f(x)=0$, $x=0$, $x=l$, $u(l,t)=At$, $u(0,t)=0$;
\item {\bfseries Inequalities}: $0\leq x \leq l$, $0\leq x \leq l^{''}$, $\epsilon >0$, $0<t<+ \infty$, $t>0$, $0<x_{0}<l$, $0\leq x\leq x_{0}$, $x_{0}\leq x \leq l$, $0<x_{0}<l$;
\item {\bfseries More general formulae}: $k_{x}$, $k_{t}$, $k_{u}$, $\bar{u}(x)=\lim\limits_{t\to +\infty} u(x,t)$, $\Phi (t)\sin \frac{\pi x}{l}$, $q(t)=Ae^{-ht}$, $u_{t}=a^{2}u_{xx}-hu$, $u_{t}=a^{2}u_{xx}-Hu+f(x,t)$, $u_{x}(0,t)-hu(0,t)=\psi_{1}(t)$, $u_{x}(l,t)+hu(l,t)=\psi_{2}(t)$, $u(x,0)=\varphi (x)$, $t \to +\infty$, $u(l,t)=A \cos \omega t$, $u_{x}(l,t)=A \cos \omega t$, $u_{x}(l,t)+hu(l,t)=A \cos \omega t$, $y=Ae^{-mx}$, $u_{t}=a^{2}u_{xx}+bu+f(x,t)$.
\end{enumerate}
As we see from the above examples, a lot of expressions have the same meaning but written in different form: for example $v_{0}$ and $v_{0}=const$, $t>0$ and $0<t<+\infty$. Therefore, as in the case of any formal language, we need fix a kind of syntax for expressions which are allowed in our model formulations.\par

\section{Examples of problem formulations}

\begin{enumerate}
\item The initial temperature of a rod $0<x<l$ is an arbitrary function of $x$. The temperatures of the ends are constant:
\begin{equation*}
u(0,t) = U_{1} = \mathrm{const}, \quad u(l,t) = U_{2} = \mathrm{const}, \quad 0<t<\infty.
\end{equation*}
A heat exchange obeying Newton's law takes place at the surface with a medium whose temperature equals $u_{0}=\mathrm{const}$. Find the temperature of the rod. Consider, in particular, the case where $U_{1}=U_{2}=0$.
\item Determine the temperature of a rod $0<x<l$ thermally insulated along its surface, if one of its ends ($x=0$) is kept at a given fixed temperature, and a given constant heat flow is maintained at the other end ($x=l$), the initial temperature being arbitrary.
\item Find the temperature of a rod $0<x<l$ thermally insulated along the surface, if heat sources of density equal to $\Phi(t)\sin\left(\dfrac{\pi x}{l}\right)$ are continuously distributed over the rod, and the initial temperature of the rod is an arbitrary function $f(x)$ and the temperature of the ends is maintained equal to zero.
\item Find the distribution of temperature in a rod $0\leq x\leq l$ thermally insulated along the surface, if the temperature of its ends is maintained to zero, and the initial temperature equals an arbitrary function $f(x)$.
\item Find the distribution of temperature in a rod $0\leq x\leq l$ thermally insulated along the surface, if the temperature of its ends is maintained to zero, and the initial temperature equals is constant $U_{0}=const$.
\item The initial temperature of a rod $0\leq x\leq l$ thermally insulated along the surface equals $U_{0}=const$, and a constant temperature is maintained at its ends 
\begin{equation*}
u(0,t)=U_{1}=const, \qquad u(l,t)=U_{2}=const, \qquad 0<t<+\infty.
\end{equation*}
Find the temperature $u(x,t)$ of the rod for $t>0$.
\item Find the distribution of temperature in a rod of length $l$ thermally insulated along the surface, if the temperature of its one end ($x=0$) is maintained at constant value $U_{1}$, and the temperature of another end ($x=l$) is zero, and the initial temperature equals an arbitrary function $\varphi(x)$.
\item Find the distribution of temperature in a rod of length $l$ thermally insulated along the surface, if the temperature of its one end ($x=0$) is zero, and the temperature of another end ($x=l$) is constant $U_{0}$, and the initial temperature is zero.
\item The initial temperature of a rod $0\leq x\leq l$ thermally insulated along the surface equals zero, and a constant temperature is maintained at its ends 
\begin{equation*}
u(0,t)=U_{1}=const, \qquad u(l,t)=U_{2}=const, \qquad 0<t<+\infty.
\end{equation*}
Find the temperature $u(x,t)$ of the rod for $t>0$.
\item Find the temperature distribution in a rod $0<x<l$ thermally insulated along the surface, if a temperature, equal to zero, is maintained at its end $x=0$ and at the end $x=l$ the temperature varies according to the law
\begin{equation*}
u(l,t)=At, \qquad A=const, \qquad 0<t<+\infty.
\end{equation*}
The initial temperature of the rod equals zero.
\end{enumerate}
To simplify the task of text recognition, we set up a basic collection of symbols which are allowed to be used in problem formulations:
\begin{enumerate}
\item {\bfseries Variables}: only $x$, $y$, $z$, $t$ are allowed as variables;
\item {\bfseries Constants}: all constants must be written by upper case letters with a sub-index (0, 1, or 2), e.g. $U_{0}$, $A_{1}$;
\item {\bfseries Functional expressions}: functions names must be written by lower case letters or by Greek letters, and arguments in parentheses must follow the function name, e.g. $\varphi(t)$, $f(x)$, $u(x,t)$, $g(x,y,t)$;
\item {\bfseries Equations}: equations must be written only with one equality sign, i.e. expressions such as $f(x)=U_{0}=const$ are not allowed; equations can be written for variables, e.g. $t=0$ or $x=l$, and for functions, e.g. $f(x)=U_{0}$, $u(0,t)$;
\item {\bfseries Inequalities}: inequalities must be written according to rules 1-4 introduced above, and can be one-sided inequality, i,e. $t>0$ or $x<L_{0}$, or two-sided inequalities, i.e. $0<x<l$, $0<t<\infty$.
\end{enumerate}\par
According to the five rules introduced above, the problem formulations can be now reformulated as follows:
\begin{enumerate}
\item The initial temperature of a rod $0<x<L_{0}$ is an arbitrary function of $x$. The temperatures of the ends are constant:
\begin{equation*}
u(0,t) = U_{1}, \quad u(L_{0},t) = U_{2}, \quad 0<t<\infty.
\end{equation*}
A heat exchange obeying Newton's law takes place at the surface with a medium whose temperature equals $U_{0}$. Find the temperature of the rod. Consider, in particular, the case where $U_{1}=0$ and $U_{2}=0$.
\item Determine the temperature of a rod $0<x<L_{0}$ thermally insulated along its surface, if one of its ends ($x=0$) is kept at a given fixed temperature, and a given constant heat flow is maintained at the other end ($x=L_{0}$), the initial temperature being arbitrary.
\item Find the temperature of a rod $0<x<L_{0}$ thermally insulated along the surface, if heat sources of density equal to $\phi(t)\sin\left(\dfrac{\pi x}{L_{0}}\right)$ are continuously distributed over the rod, and the initial temperature of the rod is an arbitrary function $f(x)$ and the temperature of the ends is maintained equal to zero.
\item Find the distribution of temperature in a rod $0\leq x\leq L_{0}$ thermally insulated along the surface, if the temperature of its ends is maintained to zero, and the initial temperature equals an arbitrary function $f(x)$.
\item Find the distribution of temperature in a rod $0\leq x\leq L_{0}$ thermally insulated along the surface, if the temperature of its ends is maintained to zero, and the initial temperature equals is $U_{0}$.
\item The initial temperature of a rod $0< x< L_{0}$ thermally insulated along the surface equals $U_{0}$, and a constant temperature is maintained at its ends 
\begin{equation*}
u(0,t)=U_{1}, \qquad u(L_{0},t)=U_{2}, \qquad 0<t<\infty.
\end{equation*}
Find the temperature $u(x,t)$ of the rod for $t>0$.
\item Find the distribution of temperature in a rod of length $L_{0}$ thermally insulated along the surface, if the temperature of its one end ($x=0$) is maintained at value $U_{1}$, and the temperature of another end ($x=L_{0}$) is zero, and the initial temperature equals an arbitrary function $\varphi(x)$.
\item Find the distribution of temperature in a rod of length $L_{0}$ thermally insulated along the surface, if the temperature of its one end ($x=0$) is zero, and the temperature of another end ($x=L_{0}$) is constant $U_{0}$, and the initial temperature is zero.
\item The initial temperature of a rod $0< x< L_{0}$ thermally insulated along the surface equals zero, and a constant temperature is maintained at its ends 
\begin{equation*}
u(0,t)=U_{1}, \qquad u(L_{0},t)=U_{2}, \qquad 0<t<\infty.
\end{equation*}
Find the temperature $u(x,t)$ of the rod for $t>0$.
\item Find the temperature distribution in a rod $0<x<L_{0}$ thermally insulated along the surface, if a temperature, equal to zero, is maintained at its end $x=0$ and at the end $x=L_{0}$ the temperature varies according to the law
\begin{equation*}
u(L_{0},t)=A_{0}t, \quad 0<t<\infty.
\end{equation*}
The initial temperature of the rod equals zero.
\end{enumerate}\par

\section{Understanding semantic of problem formulations}

In this section, we will analyse the problem formulations introduced above w.r.t. semantic of words and their connection to mathematical expressions. The goal of the modelling process is to formulate a mathematical problem, which needed to be solved, or in other words, to formulate an initial boundary value problem. For formulating the initial boundary value problem it is necessary to address the following questions:
\begin{enumerate}
\item What type of differential equation is considered (parabolic, hyperbolic, elliptic)?
\item Is the differential equation homogeneous or non-homogeneous?
\item How boundary conditions look like?
\item Is the problem static?
\item In the case of a dynamic problem, how initial conditions look like?
\end{enumerate}

\subsection{Problem 1}

{\bfseries Formulation}: The initial temperature of a rod $0<x<L_{0}$ is an arbitrary function of $x$. The temperatures of the ends are constant:
\begin{equation*}
u(0,t) = U_{1}, \quad u(L_{0},t) = U_{2}, \quad 0<t<\infty.
\end{equation*}
A heat exchange obeying Newton's law takes place at the surface with a medium whose temperature equals $U_{0}$. Find the temperature of the rod. Consider, in particular, the case where $U_{1}=0$ and $U_{2}=0$.\par
Following the five questions formulated above, let us now analyse this problem:
\begin{enumerate}
\item The differential equation of a parabolic type is considered. This is indicated by the keyword {\bfseries temperature}, because heat conduction process are typically modelled by parabolic equations. {\bfseries Important}: the parabolic heat equation will become an elliptic equation if a static process is considered.
\item The differential equation is non-homogeneous and this is indicated by the words: {\bfseries A heat exchange obeying Newton's law takes place at the surface with a medium whose temperature equals $U_{0}$}.
\item The boundary conditions are specified in the sentence: {\bfseries The temperatures of the ends are constant}.
\item The problem is time-dependent and this is indicated by words: {\bfseries The initial temperature}, because only time-dependent problems have initial conditions.
\item The initial conditions are specified in the sentence: {\bfseries The initial temperature of a rod $0<x<L_{0}$ is an arbitrary function of $x$}.
\end{enumerate}
Next step is to map the verbal description and the information we have just analysed into mathematical expressions. We will do it step by step:
\begin{itemize}
\item {\bfseries the initial temperature} $\Longrightarrow$ $u(x,0)$;
\item {\bfseries is an arbitrary function of $x$} $\Longrightarrow$ $f(x)$;
\item Thus, the first sentence {\bfseries The initial temperature of a rod $0<x<L_{0}$ is an arbitrary function of $x$.} is mapped to $u(x,0)=f(x), 0<x<L_{0}$.
\item {\bfseries The temperatures of the ends} $\Longrightarrow$ $u(0,t)$ and $u(L_{0},t)$;
\item {\bfseries A heat exchange obeying Newton's law takes place at the surface with a medium whose temperature equals $U_{0}$} $\Longrightarrow$ $-H_{0}(u(x,t)-U_{0})$;
\item {\bfseries Find the temperature of the rod.} $\Longrightarrow$ means we need to find the function $u(x,t)$.
\end{itemize}
Next, from our analysis, especially from points 1 and 4, we know that we need to consider a parabolic equation of the following form
\begin{equation*}
\frac{\partial u}{\partial t} - a^{2}\frac{\partial^{2} u}{\partial x^{2}} = 0.
\end{equation*}
Additionally, from point 2 we know that the equation is non-homogeneous, and therefore, the latter equation is rewritten now as follows
\begin{equation*}
\frac{\partial u}{\partial t} - a^{2}\frac{\partial^{2} u}{\partial x^{2}} = -H_{0}(u(x,t)-U_{0}).
\end{equation*}
Finally, summarising everything above, we get the following mathematical model:
\begin{itemize}
\item Differential equation: $\dfrac{\partial u}{\partial t} - a^{2}\dfrac{\partial^{2} u}{\partial x^{2}} = -H_{0}(u(x,t)-U_{0})$;
\item Boundary conditions: $u(0,t) = U_{1}, u(L_{0},t) = U_{2}$;
\item Initial condition: $u(x,0)=f(x)$;
\item Inequalities for variables: $0<x<L_{0}$ and $0<t<\infty$.
\end{itemize}
As the second model, we get the special case indicated by words {\bfseries Find the temperature of the rod. Consider, in particular, the case where $U_{1}=U_{2}=0$.}. This model is given then by:
\begin{itemize}
\item Differential equation: $\dfrac{\partial u}{\partial t} - a^{2}\dfrac{\partial^{2} u}{\partial x^{2}} = -H_{0}(u(x,t)-U_{0})$;
\item Boundary conditions: $u(0,t) = 0, u(L_{0},t) = 0$;
\item Initial condition: $u(x,0)=f(x)$;
\item Inequalities for variables: $0<x<L_{0}$ and $0<t<\infty$.
\end{itemize}



\subsection{Problem 2}
{\bfseries Formulation}: Determine the temperature of a rod $0<x<L_{0}$ thermally insulated along its surface, if one of its ends ($x=0$) is kept at a given fixed temperature, and a given constant heat flow is maintained at the other end ($x=L_{0}$), the initial temperature being arbitrary.\par
Following again the five questions, for this problem we get:
\begin{enumerate}
\item The differential equation of a parabolic type is considered. This is indicated by the keyword {\bfseries temperature}.
\item The differential equation is homogeneous, because the problem formulation does not contain the words: {\bfseries heat source}, {\bfseries Newton's law}, and {\bfseries Fourier's law}.
\item The boundary conditions are specified in the sentence: {\bfseries if one of its ends ($x=0$) is kept at a given fixed temperature, and a given constant heat flow is maintained at the other end ($x=L_{0}$)}.
\item The problem is time-dependent and this is indicated by words: {\bfseries The initial temperature}.
\item The initial conditions are specified by words: {\bfseries the initial temperature being arbitrary}.
\end{enumerate}
Next step is to map the verbal description and the information we have just analysed into mathematical expressions:
\begin{itemize}
\item {\bfseries Determine the temperature of a rod} $\Longrightarrow$ means we need to find the function $u(x,t)$;
\item {\bfseries if one of its ends ($x=0$) is kept at a given fixed temperature} $\Longrightarrow$ $u(0,t)=U_{0}$;
\item {\bfseries a given constant heat flow is maintained at the other end ($x=l$)} $\Longrightarrow$ $\lambda \dfrac{\partial u}{\partial x}(L_{0},t)=U_{1}$;
\item {\bfseries the initial temperature being arbitrary} is mapped to $u(x,0)=f(x), 0<x<L_{0}$.
\end{itemize}
Next, from our analysis, especially from points 1, 2, and 4, we know that we need to consider a parabolic equation of the following form
\begin{equation*}
\frac{\partial u}{\partial t} - a^{2}\frac{\partial^{2} u}{\partial x^{2}} = 0.
\end{equation*}
Finally, summarising everything above, we get the following mathematical model:
\begin{itemize}
\item Differential equation: $\dfrac{\partial u}{\partial t} = a^{2}\dfrac{\partial^{2} u}{\partial x^{2}}$;
\item Boundary conditions: $u(0,t)=U_{0}, \lambda \dfrac{\partial u}{\partial x}(L_{0},t)=U_{1}$;
\item Initial condition: $u(x,0)=f(x)$;
\item Inequalities for variables: $0<x<L_{0}$ and $0<t<\infty$.
\end{itemize}

\subsection{Problem 3}

{\bfseries Formulation}: Find the temperature of a rod $0<x<L_{0}$ thermally insulated along the surface, if heat sources of density equal to $\phi(t)\sin\left(\dfrac{\pi x}{L_{0}}\right)$ are continuously distributed over the rod, and the initial temperature of the rod is an arbitrary function $f(x)$ and the temperature of the ends is maintained equal to zero.
\begin{enumerate}
\item The differential equation of a parabolic type is considered. This is indicated by the keyword {\bfseries temperature}.
\item The differential equation is non-homogeneous, because the problem formulation contains the words: {\bfseries heat sources}.
\item The boundary conditions are specified in the sentence: {\bfseries the temperature of the ends is maintained equal to zero}.
\item The problem is time-dependent and this is indicated by words: {\bfseries The initial temperature}.
\item The initial conditions are specified by words: {\bfseries the initial temperature of the rod is an arbitrary function $f(x)$}.
\end{enumerate}
Next step is to map the verbal description and the information we have just analysed into mathematical expressions:
\begin{itemize}
\item {\bfseries Find the temperature of a rod} $\Longrightarrow$ means we need to find the function $u(x,t)$;
\item {\bfseries if heat sources of density equal to $\phi(t)\sin\left(\dfrac{\pi x}{L_{0}}\right)$ are continuously distributed over the rod} $\Longrightarrow$ means that the right-hand side of the equation is $\phi(t)\sin\left(\dfrac{\pi x}{L_{0}}\right)$;
\item {\bfseries the initial temperature of the rod is an arbitrary function $f(x)$} $\Longrightarrow$ $u(x,0)=f(x), 0<x<L_{0}$;
\item {\bfseries the temperature of the ends is maintained equal to zero} $\Longrightarrow$ $u(0,t)=0$ and $u(L_{0},t)=0$.
\end{itemize}
Finally, summarising everything above, we get the following mathematical model:
\begin{itemize}
\item Differential equation: $\dfrac{\partial u}{\partial t} - a^{2}\dfrac{\partial^{2} u}{\partial x^{2}}=\phi(t)\sin\left(\dfrac{\pi x}{L_{0}}\right)$;
\item Boundary conditions: $u(0,t)=0, u(L_{0},t)=0$;
\item Initial condition: $u(x,0)=f(x)$;
\item Inequalities for variables: $0<x<L_{0}$ and $0<t<\infty$.
\end{itemize}

\subsection{Problem 4}

{\bfseries Formulation}: Find the distribution of temperature in a rod $0\leq x\leq L_{0}$ thermally insulated along the surface, if the temperature of its ends is maintained to zero, and the initial temperature equals an arbitrary function $f(x)$.
\begin{enumerate}
\item The differential equation of a parabolic type is considered. This is indicated by the keyword {\bfseries temperature}.
\item The differential equation is homogeneous, because the problem formulation does not contain the words: {\bfseries heat source}, {\bfseries Newton's law}, and {\bfseries Fourier's law}.
\item The boundary conditions are specified in the sentence: {\bfseries the temperature of its ends is maintained to zero}.
\item The problem is time-dependent and this is indicated by words: {\bfseries the initial temperature}.
\item The initial conditions are specified by words: {\bfseries the initial temperature equals an arbitrary function $f(x)$}.
\end{enumerate}
Next step is to map the verbal description and the information we have just analysed into mathematical expressions:
\begin{itemize}
\item {\bfseries Find the distribution of temperature in a rod} $\Longrightarrow$ means we need to find the function $u(x,t)$;
\item {\bfseries the temperature of the ends is maintained to zero} $\Longrightarrow$ $u(0,t)=0$ and $u(L_{0},t)=0$
\item {\bfseries the initial temperature equals an arbitrary function $f(x)$} $\Longrightarrow$ $u(x,0)=f(x), 0<x<L_{0}$.
\end{itemize}
Finally, summarising everything above, we get the following mathematical model:
\begin{itemize}
\item Differential equation: $\dfrac{\partial u}{\partial t} - a^{2}\dfrac{\partial^{2} u}{\partial x^{2}}=0$;
\item Boundary conditions: $u(0,t)=0, u(L_{0},t)=0$;
\item Initial condition: $u(x,0)=f(x)$;
\item Inequalities for variables: $0<x<L_{0}$ and $0<t<\infty$.
\end{itemize}

\subsection{Problem 5}

{\bfseries Formulation}: Find the distribution of temperature in a rod $0\leq x\leq L_{0}$ thermally insulated along the surface, if the temperature of its ends is maintained to zero, and the initial temperature equals is $U_{0}$.
\begin{enumerate}
\item The differential equation of a parabolic type is considered. This is indicated by the keyword {\bfseries temperature}.
\item The differential equation is homogeneous, because the problem formulation does not contain the words: {\bfseries heat source}, {\bfseries Newton's law}, and {\bfseries Fourier's law}.
\item The boundary conditions are specified in the sentence: {\bfseries the temperature of its ends is maintained to zero}.
\item The problem is time-dependent and this is indicated by words: {\bfseries the initial temperature}.
\item The initial conditions are specified by words: {\bfseries the initial temperature equals is $U_{0}$}.
\end{enumerate}
Next step is to map the verbal description and the information we have just analysed into mathematical expressions:
\begin{itemize}
\item {\bfseries Find the distribution of temperature in a rod} $\Longrightarrow$ means we need to find the function $u(x,t)$;
\item {\bfseries the temperature of the ends is maintained to zero} $\Longrightarrow$ $u(0,t)=0$ and $u(L_{0},t)=0$
\item {\bfseries the initial temperature equals is $U_{0}$} $\Longrightarrow$ $u(x,0)=U_{0}, 0<x<L_{0}$.
\end{itemize}
Finally, summarising everything above, we get the following mathematical model:
\begin{itemize}
\item Differential equation: $\dfrac{\partial u}{\partial t} - a^{2}\dfrac{\partial^{2} u}{\partial x^{2}}=0$;
\item Boundary conditions: $u(0,t)=0, u(L_{0},t)=0$;
\item Initial condition: $u(x,0)=U_{0}$;
\item Inequalities for variables: $0<x<L_{0}$ and $0<t<\infty$.
\end{itemize}

\subsection{Problem 6}

{\bfseries Formulation}: The initial temperature of a rod $0< x< L_{0}$ thermally insulated along the surface equals $U_{0}$, and a constant temperature is maintained at its ends 
\begin{equation*}
u(0,t)=U_{1}, \qquad u(L_{0},t)=U_{2}, \qquad 0<t<\infty.
\end{equation*}
Find the temperature $u(x,t)$ of the rod for $t>0$.
\begin{enumerate}
\item The differential equation of a parabolic type is considered. This is indicated by the keyword {\bfseries temperature}.
\item The differential equation is homogeneous, because the problem formulation does not contain the words: {\bfseries heat source}, {\bfseries Newton's law}, and {\bfseries Fourier's law}.
\item The boundary conditions are specified in the sentence: {\bfseries a constant temperature is maintained at its ends}.
\item The problem is time-dependent and this is indicated by words: {\bfseries the initial temperature}.
\item The initial conditions are specified by words: {\bfseries The initial temperature of a rod $0< x< L_{0}$ thermally insulated along the surface equals $U_{0}$}.
\end{enumerate}
Next step is to map the verbal description and the information we have just analysed into mathematical expressions:
\begin{itemize}
\item {\bfseries The initial temperature of a rod $0< x< L_{0}$ thermally insulated along the surface equals $U_{0}$} $\Longrightarrow$ $u(x,0)=U_{0}, 0<x<L_{0}$;
\item {\bfseries  a constant temperature is maintained at its ends} $\Longrightarrow$ $u(0,t)=U_{1}$ and $u(L_{0},t)=U_{2}$;
\item {\bfseries Find the temperature} $\Longrightarrow$ means we need to find the function $u(x,t)$.
\end{itemize}
Finally, summarising everything above, we get the following mathematical model:
\begin{itemize}
\item Differential equation: $\dfrac{\partial u}{\partial t} - a^{2}\dfrac{\partial^{2} u}{\partial x^{2}}=0$;
\item Boundary conditions: $u(0,t)=U_{1}$, $u(L_{0},t)=U_{2}$;
\item Initial condition: $u(x,0)=U_{0}$;
\item Inequalities for variables: $0<x<L_{0}$ and $0<t<\infty$.
\end{itemize}

\subsection{Problem 7}

{\bfseries Formulation}: Find the distribution of temperature in a rod of length $L_{0}$ thermally insulated along the surface, if the temperature of its one end ($x=0$) is maintained at value $U_{1}$, and the temperature of another end ($x=L_{0}$) is zero, and the initial temperature equals an arbitrary function $\varphi(x)$.
\begin{enumerate}
\item The differential equation of a parabolic type is considered. This is indicated by the keyword {\bfseries temperature}.
\item The differential equation is homogeneous, because the problem formulation does not contain the words: {\bfseries heat source}, {\bfseries Newton's law}, and {\bfseries Fourier's law}.
\item The boundary conditions are specified in the sentence: {\bfseries if the temperature of its one end ($x=0$) is maintained at value $U_{1}$, and the temperature of another end ($x=L_{0}$) is zero}.
\item The problem is time-dependent and this is indicated by words: {\bfseries the initial temperature}.
\item The initial conditions are specified by words: {\bfseries the initial temperature equals an arbitrary function $\varphi(x)$}.
\end{enumerate}
Next step is to map the verbal description and the information we have just analysed into mathematical expressions:
\begin{itemize}
\item {\bfseries Find the distribution of temperature in a rod of length $L_{0}$ thermally insulated along the surface} $\Longrightarrow$ means we need to find the function $u(x,t)$;
\item {\bfseries the temperature of its one end ($x=0$) is maintained at value $U_{1}$, and the temperature of another end ($x=L_{0}$) is zero} $\Longrightarrow$ $u(0,t)=U_{1}$ and $u(L_{0},t)=0$
\item {\bfseries the initial temperature equals an arbitrary function $\varphi(x)$} $\Longrightarrow$ $u(x,0)=\varphi(x), 0<x<L_{0}$.
\end{itemize}
Finally, summarising everything above, we get the following mathematical model:
\begin{itemize}
\item Differential equation: $\dfrac{\partial u}{\partial t} - a^{2}\dfrac{\partial^{2} u}{\partial x^{2}}=0$;
\item Boundary conditions: $u(0,t)=U_{1}, u(L_{0},t)=0$;
\item Initial condition: $u(x,0)=\varphi(x)$;
\item Inequalities for variables: $0<x<L_{0}$ and $0<t<\infty$.
\end{itemize}

\subsection{Problem 8}

{\bfseries Formulation}: Find the distribution of temperature in a rod of length $L_{0}$ thermally insulated along the surface, if the temperature of its one end ($x=0$) is zero, and the temperature of another end ($x=L_{0}$) is constant $U_{0}$, and the initial temperature is zero.
\begin{enumerate}
\item The differential equation of a parabolic type is considered. This is indicated by the keyword {\bfseries temperature}.
\item The differential equation is homogeneous, because the problem formulation does not contain the words: {\bfseries heat source}, {\bfseries Newton's law}, and {\bfseries Fourier's law}.
\item The boundary conditions are specified in the sentence: {\bfseries if the temperature of its one end ($x=0$) is zero, and the temperature of another end ($x=L_{0}$) is constant $U_{0}$}.
\item The problem is time-dependent and this is indicated by words: {\bfseries the initial temperature}.
\item The initial conditions are specified by words: {\bfseries the initial temperature is zero}.
\end{enumerate}
Next step is to map the verbal description and the information we have just analysed into mathematical expressions:
\begin{itemize}
\item {\bfseries Find the distribution of temperature in a rod of length $L_{0}$ thermally insulated along the surface} $\Longrightarrow$ means we need to find the function $u(x,t)$;
\item {\bfseries if the temperature of its one end ($x=0$) is zero, and the temperature of another end ($x=L_{0}$) is constant $U_{0}$} $\Longrightarrow$ $u(0,t)=0$ and $u(L_{0},t)=U_{0}$
\item {\bfseries the initial temperature is zero} $\Longrightarrow$ $u(x,0)=0, 0<x<L_{0}$.
\end{itemize}
Finally, summarising everything above, we get the following mathematical model:
\begin{itemize}
\item Differential equation: $\dfrac{\partial u}{\partial t} - a^{2}\dfrac{\partial^{2} u}{\partial x^{2}}=0$;
\item Boundary conditions: $u(0,t)=0, u(L_{0},t)=U_{0}$;
\item Initial condition: $u(x,0)=0$;
\item Inequalities for variables: $0<x<L_{0}$ and $0<t<\infty$.
\end{itemize}

\subsection{Problem 9}

{\bfseries Formulation}: The initial temperature of a rod $0< x< L_{0}$ thermally insulated along the surface equals zero, and a constant temperature is maintained at its ends 
\begin{equation*}
u(0,t)=U_{1}, \qquad u(L_{0},t)=U_{2}, \qquad 0<t<\infty.
\end{equation*}
Find the temperature $u(x,t)$ of the rod for $t>0$.
\begin{enumerate}
\item The differential equation of a parabolic type is considered. This is indicated by the keyword {\bfseries temperature}.
\item The differential equation is homogeneous, because the problem formulation does not contain the words: {\bfseries heat source}, {\bfseries Newton's law}, and {\bfseries Fourier's law}.
\item The boundary conditions are specified in the sentence: {\bfseries a constant temperature is maintained at its ends }.
\item The problem is time-dependent and this is indicated by words: {\bfseries the initial temperature}.
\item The initial conditions are specified by words: {\bfseries The initial temperature of a rod $0< x< L_{0}$ thermally insulated along the surface equals zero}.
\end{enumerate}
Next step is to map the verbal description and the information we have just analysed into mathematical expressions:
\begin{itemize}
\item {\bfseries The initial temperature of a rod $0< x< L_{0}$ thermally insulated along the surface equals zero} $\Longrightarrow$ $u(x,0)=0, 0<x<L_{0}$;
\item {\bfseries  a constant temperature is maintained at its ends} $\Longrightarrow$ $u(0,t)=U_{1}$ and $u(L_{0},t)=U_{2}$;
\item {\bfseries Find the temperature} $\Longrightarrow$ means we need to find the function $u(x,t)$.
\end{itemize}
Finally, summarising everything above, we get the following mathematical model:
\begin{itemize}
\item Differential equation: $\dfrac{\partial u}{\partial t} - a^{2}\dfrac{\partial^{2} u}{\partial x^{2}}=0$;
\item Boundary conditions: $u(0,t)=U_{1}$, $u(L_{0},t)=U_{2}$;
\item Initial condition: $u(x,0)=0$;
\item Inequalities for variables: $0<x<L_{0}$ and $0<t<\infty$.
\end{itemize}

\subsection{Problem 10}

{\bfseries Formulation}: Find the temperature distribution in a rod $0<x<L_{0}$ thermally insulated along the surface, if a temperature, equal to zero, is maintained at its end $x=0$ and at the end $x=L_{0}$ the temperature varies according to the law
\begin{equation*}
u(L_{0},t)=A_{0}t, \quad 0<t<\infty.
\end{equation*}
The initial temperature of the rod equals zero.
\begin{enumerate}
\item The differential equation of a parabolic type is considered. This is indicated by the keyword {\bfseries temperature}.
\item The differential equation is homogeneous, because the problem formulation does not contain the words: {\bfseries heat source}, {\bfseries Newton's law}, and {\bfseries Fourier's law}.
\item The boundary conditions are specified in the sentence: {\bfseries if a temperature, equal to zero, is maintained at its end $x=0$ and at the end $x=L_{0}$ the temperature varies according to the law}.
\item The problem is time-dependent and this is indicated by words: {\bfseries the initial temperature}.
\item The initial conditions are specified by words: {\bfseries The initial temperature of the rod equals zero.}.
\end{enumerate}
Next step is to map the verbal description and the information we have just analysed into mathematical expressions:
\begin{itemize}
\item {\bfseries Find the temperature distribution in a rod $0<x<L_{0}$ thermally insulated along the surface} $\Longrightarrow$ means we need to find the function $u(x,t)$;
\item {\bfseries  if a temperature, equal to zero, is maintained at its end $x=0$} $\Longrightarrow$ $u(0,t)=0$;
\item {\bfseries  and at the end $x=L_{0}$ the temperature varies according to the law $u(L_{0},t)=A_{0}t$} $\Longrightarrow$ $u(L_{0},t)=A_{0}t$;
\item {\bfseries The initial temperature of the rod equals zero.} $\Longrightarrow$ $u(x,0)=0, 0<x<L_{0}$;
\end{itemize}
Finally, summarising everything above, we get the following mathematical model:
\begin{itemize}
\item Differential equation: $\dfrac{\partial u}{\partial t} - a^{2}\dfrac{\partial^{2} u}{\partial x^{2}}=0$;
\item Boundary conditions: $u(0,t)=0$, $u(L_{0},t)=A_{0}t$;
\item Initial condition: $u(x,0)=0$;
\item Inequalities for variables: $0<x<L_{0}$ and $0<t<\infty$.
\end{itemize}
