\documentclass[12pt]{report}
\usepackage[utf8]{inputenc}
\usepackage{amsmath}
\usepackage{graphicx}
\usepackage{appendix}
\usepackage{amssymb}
\usepackage{amsmath}
\usepackage{amsthm}
\usepackage{tikz}
\usepackage{textcomp}
\graphicspath{{images/}}
\usepackage[a4paper,width=150mm,top=25mm,bottom=25mm]{geometry}
\usepackage{fancyhdr}
\usepackage{epigraph}
\usepackage[style=authoryear]{biblatex}
\usepackage{listings}
\usepackage{tikz}
\usepackage[utf8]{inputenc}
\usepackage{tcolorbox}


\usepackage{listings}
\usepackage{xcolor}



\usetikzlibrary{arrows.meta,
                chains,
                positioning,
                shapes.geometric
                }

\pagestyle{fancy}
\fancyhead{}
\fancyhead[]{}
\addbibresource{references.bib}

\title{Generating mathematical models by help of deep learning}
\begin{document}
\begin{titlepage}
	\centering
	\includegraphics[width=0.5\textwidth]{logo.png}\par\vspace{1cm}
	{\scshape\LARGE Bauhaus Universität Weimar \par}
	\vspace{1cm}
	{\scshape\Large Special project\par}
	\vspace{1.5cm}
	{\huge\bfseries Generating mathematical models by help of deep learning\par}
	\vspace{2cm}
	{\Large\itshape Karl Philipp Killian\par}
	{\Large\itshape Yohannes Gidey Zegeye\par}
	\vfill
	supervised by\par
	Dr. rer. nat. Dmitrii  \textsc{Legatiuk}
	\par
    Dipl.-Ing. Anastasiia  \textsc{Legatiuk}
	\vfill

% Bottom of the page
	{\large \today\par}
\end{titlepage}

\chapter*{Declaration}
We, the authors of this report, confirm that the work for the following report with the title: “Generating mathematical models by help of deep learning” was solely undertaken by
ourselves and that no help was provided by other sources as those allowed. All sections of the report that uses quotes or describe an argument or concept developed by another author have been referenced, including all secondary literature used, to show that this material has been adopted to support this report.

\vspace*{4em}\noindent
\hfill%
\begin{tabular}[t]{c}
  \rule{10em}{0.4pt}\\ Karl Philipp Killian
\end{tabular}%
\hfill%
\begin{tabular}[t]{c}
  \rule{10em}{0.4pt}\\ Yohannes Gidey Zegeye
\end{tabular}%
\hfill\strut


\chapter*{Abstract}
In this report the students developed an approach for solving mathematical problems related to engineering by the help of deep learning. The goal is to extract boundary conditions and trigger words that determine type of word problems. The difficulty consists in finding an appropriate software library which can handle input of mathematical expressions within normal text. After a discovery phase we chose spaCy and its builtin NER system for information extraction. Owing the small amount of data a preprocessing is done, which provides a stronger bias. However, our results are a good demonstration what state-of-the-art CNN in the field of NLP is capable of achieving and are comparable to other research work that have been done. 

\tableofcontents
\chapter{Introduction}
\epigraph{Success in creating AI would be the biggest event in human history.}{\textit{Stephen Hawking}}

In the wake of more powerful computers, lower memory costs and broad investments by tech giants, artificial intelligence has been growing in popularity for some time.  In the Deep Learning subcategory, major advances have been made recently, which include facial and speech recognition as well as autonomous driving. Furthermore, in 2016, the computer progam AlphaGo succeeded in beating the world's best Go player Lee Sedol for the first time.\\

Nevertheless, there are areas in which the success of neural networks is limited. One of these topics is the processing of mathematical problems in the fields of engineering or physics. Only a few papers tried to tackle this problem \parencite{wang2017deep} and Wong considered solving linear algebraic one-variable tasks. Wang et al. showed that sequence to sequence (seq2seq) outperforms state-of-the-art statistical approaches. The outcome of these reports make one optimistic that neural networks are a very promising method of choice for solving these tasks. 

For higher mathematical problems no papers cannot be found. Lample and Charton from Facebook considered this typ of problem, but focused on solving mathematical equations as an direct input by help of deep learning. \\

In this report, we consider mathematical expressions appearing in engineering problem formulations. With the help of Natural Language Processing (NLP) the given problem in text form shall be translated to resulting equations.

\chapter{Related work}
Understanding semantics of a text has been a huge research area in natural language processing. Nowadays many of research and industry work on common applications such as customer services or chatbots can be found. From understanding the context to turning instructions into actions, NLP is essential state-of-the-art technology.

Our focus is semantic analysis for information extraction and in that realm, a few works such as  \parencite{kushman2014learning}, \parencite{chambers2011template} and \parencite{chen2020mapping} have been done. They all dealed with finding a meaningful representation of information from an implicit natural language.

Although \parencite{kushman2014learning} work is solving algebraic word problems than formulating the equations, that is one part of it and it incorporates grounding equations using pattern templates. They learn mapping from a wide range of templates, which are induced from the training set where each example in the set have slots to be filled with values from the sentences and unknowns filled with nouns. This method applies to only algebraic word problems with the four arithmetic operators $(x,-,+,/)$ and relies on the word problem structure similarity and requires to draft a number of templates to handle different word problem types. 

\parencite{koncel2015parsing}  also introduces a different approach for the algebraic word problem equation grounding. ALGES, which is given an unlabeled word problem with its solution, maps word problems into equation trees by matching a text to arithmetic operators, and combining noun phrases to operand nodes.

\parencite{roy2016equation} presents a mapping mechanism where a vocabulary of trigger words for variables, quantities are mapped to one or two variables, and are combined with operands to form a binary tree. For each word problem, an algorithm learns the possible trigger output which will be matched to a lexicon of equation representation. However, this also has a limitation of working with single sentences. 

While most of the mentioned works is mostly rule based approaches, \parencite{huang-etal-2018-neural} used seq2seq model to formulate to solve equations with a recursive neural network. In seq2seq models recognized numbers get mapped to a list of number tokens, which afterwards is solved.

\parencite{shi2015automatically}, introduces dolphine language (DOL), as a semantic representation of the text. Further a semantic parser is used to transfer the DOL tree into mathematical expression. From the tree representation a reasoning module is used to map it into math expression.  

Perhaps a more closely related work \parencite{hosseini2014learning}, introduces an algorithm ARIS that will ground the a word problem and solves the equation. Much like the previous papers it deals with arithmetic word problems. Furthermore it breaks down a word problem into fragments of phrases, verbs, entities and containers which will be the basis for solving the problem. While ARIS uses a classifier to label the different category of verbs which is used to emphasize the impact of a chunk of phrase or word extract, the more information extraction is done by a Stanford’s dependency parser, named entity recognizer which is very similar to our approach. ARIS deals with multi sentences word problems but still limited to arithmetic formulations.\\
\par

\begin{tcolorbox}
The initial temperature of a rod $0<x<l$ is an arbitrary function of $x$. The temperatures of the ends are constant:
\begin{equation*}
u(0,t) = U_{1} = \mathrm{const}, \quad u(l,t) = U_{2} = \mathrm{const}, \quad 0<t<\infty.
\end{equation*}
A heat exchange obeying Newton's law takes place at the surface with a medium whose temperature equals $u_{0}=\mathrm{const}$. Find the temperature of the rod. Consider, in particular, the case where $U_{1}=U_{2}=0$.
\end{tcolorbox}\\
As depicted in the box our approach deals majorly with a different domain and complex word problems that combine multiple variables and mathematical representations, and having a of mix pattern based mapping to deep learning could suffice to our problem. However, NLP is one of the most important technology to solve our problem, therefore a more detailed look is needed.


\chapter{Preliminaries}
Nearly everyone has heard the terms machine learning (ML), artificial intelligence (AI) and deep learning (DL). However, it can be difficult to classify these and bring them in a correct context with NLP. To ensure a uniform understanding, these terms are defined. 
AI is the overarching discipline that covers anything related to making machines smart. Whether it is a robot, a refrigerator, a car, or a software application, if you are making them smart, then it is AI. ML is commonly used alongside AI but they are not the same thing. ML is a subset of AI, it refers to systems that can learn by themselves. Systems that get smarter and smarter over time without human intervention. DL is ML but applied to large complex data sets. Most AI work now involves ML because intelligent behavior requires considerable knowledge, and learning is the easiest way to get that knowledge. The image \ref{fig:overview} captures the relationship between AI, ML, and DL.

\begin{figure}[h]
    \centering
    \includegraphics[width=0.8\textwidth]{images/overview.png}
    \caption{Euler diagram}
    \label{fig:overview}
\end{figure}

\section{NLP}
Natural Language Processing (NLP) attempts to capture natural language and process it in a computer-based way using rules and algorithms. NLP uses a variety of methods and results from the field of linguistics and combines them with computer science and artificial intelligence \parencite{liddy2001natural}.
For a better understanding figure \ref{fig:phases} defines phases of analysis in processing natural language. The phases can also be considered as individual tasks. It is sufficient to consider for this problem the first four stages \parencite{indurkhya2010handbook}.
  
\begin{figure}[h]
    \centering
    \resizebox{1\textwidth}{!}{
    \input{images/flowchart}
    }
    \caption{Phases of analysis in processing natural language}
    \label{fig:phases}
\end{figure}

\begin{description}

\item[Tokenization]\hfill \\
Tokenization and putting sentences into certain segments is a very important step, as you cannot take it for granted that text is well formatted and structured, for example in the case of delimiters. Furthermore, in some languages like Chinese and Japanese there is the difficulty of not having an easy space limiter as in the English language. However, text preprocessing is a crucial step for the following stages and can be challenging depending on the given input.

\item[Lexical analysis]\hfill \\
Now we have given individual words, which are in need of closer inspection. This tasks touches the domain of morphology. The words will be classified by their rank of meaning and independence to other words. By decomposing the words a large amount of rules have to be taken into account. Conversion is one of these rules, which allows to create a new word from an existing word by not changing its form. This word formation works for nouns to verbs and vice versa. Also, conversions from adjectives to nouns are possible, for example, pet $\rightarrow$ to pet; innocent $\rightarrow$ the innocent.

\item[Syntactic analysis]\hfill \\
In this part the check if the sentence is conform with formal grammar is done. The sentence is parsed by getting information about the words which include certain meaning, depending on their sequence or structure. Though a sentence can be syntactically correct, it can be semantically incorrect. For example, "cows flow supremely" is grammatically valid but it is senseless.

\item[Semantic analysis]\hfill \\
Semantic analysis is the process of understanding the meaning and interpretation of words, signs and sentence structure. This lets computers partly understand natural language the way humans do. Semantic analysis is one of the toughest parts of NLP and it is not fully solved yet. Also our solving approach with Named-entity recognition is located here. It is briefly described in the chapter 4.3.1.
\end{description}

For solving these tasks symbolic, statistical and nowadays neural network methods are used. This is due to the fact that the computing performance significantly increased over the time. Nevertheless the old methods are still used today. For example in the tokenization area symbolic methods are still en vogue.


\chapter{Approach}
Our approach starts with preparing the word problem data sets. From our data we need to extract information that will help us formulate the mathematical model from the given text. There are different ways we can do it but the major ones are using a deep learning algorithm or by using a pattern matching using regular expression for instance. 
With regex (rule based) we can use different patterns to extract different information, but this is particularly hard because it is difficult to have a single pattern or even multiple patterns that can satisfy every condition or info that we want. 
Thus, we will use deep learning and packages with different models to handle post tagging and custom NER to train a model that can be used to extract required information. 

\section{Data understanding} 
Our task is to formalize a verbal description of engineering problem into mathematical model using expressions and equations. As such we need to understand the type of text available to prepare it to the modelling stage. 
The first step is the data acquisition which was gathered from collection of problems from \parencite{budak1965collection}.
We consider problems in physical phenomenon such as heat conduction, diffusion, the propagation of electromagnetic waves in conducting media and the motion of viscous fluid. 
Budak presents a series of boundary value problems where the physical process can be described by functions of two independent variables, time and coordinates.  
The problems are of a different kind, ones of hyperbolic or parabolic with each determining the differential equation representing them. 
From visual inspection of the data we can see that a pattern of information such as inequality equations, similar wording describing a condition that is useful for the formulation of equations.

To state the boundary-value problem, corresponding to a given physical problem, means to choose a function describing the physical process, and then
\begin{itemize}
    \item derive the differential equation of this function,
    \item establish boundary conditions for it,
    \item formulate the initial conditions.
\end{itemize}

\newpage
Unfortunately finding a data set of similar problem statements is not easy possibly because the area of the problems are a little advanced or not commonly solved by the online community as compared to an algebraic word problems which is widely available on the web such as \textbf{\textit{algebra.com}}, \textbf{\textit{yahoo.answers.com}} which can be scraped into data frame for processing. Thus relaying on hard text books to get data of appropriate discrepancy and not a generically similar problems and on that basis we have 24 data sets word problems as can be seen in appendix \ref{problems}. 

\section{Data preparation} 
 As a start to the data preprocessing stage, we transformed the text data into a suitable format for the modeling. 
 \begin{figure}[bht]
    \centering
    \includegraphics[width=1\textwidth]{images/pipeline.png}
    \caption{Deep model pipeline}
    \label{fig:4 complete model procesies}
\end{figure}
\subsection{Annotation and cleaning data} 
Semantic annotation is the task of annotating or labeling various concepts or tags within text, from a predefined categories such as people, objects, or company names. Machine learning models use semantically annotated data to learn how to categorize new concepts in new texts. 
For our data we choose three labels initial condition, function and boundary condition to help train our formalization model. 

Although there are tools to be used for annotation purposes, we used a manual method since the data sets are small in number. In our we cleaned the data by removing unnecessary spaces, commas and parenthesis so it can be easier for training. This process is discribed in the appendix \ref{problems}.                                   
\subsection{Exploratory data analysis (EDA)}
\subsubsection{Word embedding} 
When dealing with natural language processing, the data is often in a text format. One way of representing the text is using a word embedding technique where individual words are represented as real values vectors in a predefined vector space. Word embedding are N-dimensional vectors that try to capture word meaning and context in their values. Each word is mapped to one vector and the vector values are learned in a way that resembles a neutral network. This is done by algorithms that cluster similar words together and projects it in multiple dimension of vectors. 
For example:
\begin{figure}[hbt!]
    \centering
    \includegraphics[width=1\textwidth]{images/word embedding.PNG}
    \caption{word embedding}
    \label{fig:word rmbedding}
\end{figure}

  There are a lot of word embedding algorithms developed with Word2Vec being the most famous. GloVe and fasttext are also other algorithms in the industry. spaCy uses word2vec on its pretrained English model. Word2vec takes as its input a large corpus of text and produces a vector space with each unique word in the corpus being assigned a corresponding vector in the space.  
Last the data is splitted into training and test data by a common ratio 80/20.
\newpage
\section{Modelling} 
In NLP most tasks such as tagger, parser, text categorizer and many other functionalities are product of statistical models. To extract the information embedded in the texts we will use a technique called Named-entity recognition. 
\begin{figure}[hbt!]
    \centering
    \includegraphics[width=1\textwidth]{images/spacys_NER_model.png}
    \caption{NER model of spaCy}
    \label{fig:NER}
\end{figure}
 
\subsection{Named-entity recognition (NER)}
NER is sub task of information extraction that seek to locate and classify named entities mentioned in unstructured text into predefined categories such as person names, organization,location, medical codes, time expression,quantities, monetary values, etc.  
\begin{figure}[hbt!]
    \centering
    \includegraphics[width=1\textwidth]{images/NER_annotation.png}
    \caption{NER labeling}
    \label{fig:annotationl}
\end{figure} 
With NER we can identify the meaning of "Apple" to mean the company apple and not the fruit which can be a node in a knowledge base outside of the linguistic network. 
Similar to the other NLP tasks mentioned above, this also is done by statistical Machine leaning algorithms such as RNN, LSTM, random forest or feed forward neural network. 

\subsubsection{spaCys NER model} 
Instead of looking at the labeled text and its meaning NER looks into the surrounding words and try to figure out what the labeled word mean. spaCy uses a new deep learning concept called \textbf{\textit{embed, encode, attend, predict.}}

\begin{itemize}
    \item Embed is a process of turning a text or sparse, binary vectors into shorter, dense vectors using a bloom embedding algorithm or hash embed. This takes a different approach to the common embedding representations.
    \item Encode is a process that converts a sequence of word vectors into a sentence matrix where each row represents a meaning of each token in the context of the rest of the sentence there by capturing the semantic representation by using a convolutional neural network which are several layers of convolutions, which is a linear operation that involves the multiplication of a set of weights with inputs much like traditional neural network, with non linear activation functions applied to the outputs.
    \item Attend is a process of reducing the sentence matrix representation of the encoding output into a single vector using the attention mechanism which is a technique that mimics cognitive attention which means the network focuses its memory into important parts of the data it processes. 
    \item Predict is a mechanism of predicting given label with a simple multi layer percepton given an input.
    
\end{itemize}
With the above four principles spaCys NER can be used to extract information from our math word problems. But since the NER model english model is trained on a common conversational English we can customize the model by training it with our own corpus. 
\begin{figure}[h]
    \centering
\includegraphics[width=1\textwidth]{images/spacy_NLP_pipeline.png}
    \caption{NLP pipepline for spaCy}
    \label{fig:NLP pipelinel}
\end{figure} 
Using spaCy we will custom build the NER model and leave the other component models such as tokenizer, tagger and parser intact to get the document object from the input text. Figure \ref{fig:NLP pipelinel} helps to explain the processing pipeline.
When formulating the mathematical equations what we need from the model is semantically correct inputs or know value. Thus we have boundary conditions and initial conditions from the semantic.
\definecolor{codegreen}{rgb}{0,0.6,0}
\definecolor{codegray}{rgb}{0.5,0.5,0.5}
\definecolor{codepurple}{rgb}{0.58,0,0.82}
\definecolor{backcolour}{rgb}{0.95,0.95,0.92}

\lstdefinestyle{mystyle}{
    backgroundcolor=\color{backcolour},   
    commentstyle=\color{codegreen},
    keywordstyle=\color{magenta},
    numberstyle=\tiny\color{codegray},
    stringstyle=\color{codepurple},
    basicstyle=\ttfamily\footnotesize,
    breakatwhitespace=false,         
    breaklines=true,                 
    captionpos=b,                    
    keepspaces=true,                 
    numbers=left,                    
    numbersep=5pt,                  
    showspaces=false,                
    showstringspaces=false,
    showtabs=false,                  
    tabsize=2
}
\lstset{style=mystyle}
\begin{lstlisting}[language=python]

doc=nlp("""The initial temperature of a rod 0 < x < L0 thermally 
insulated along the surface equals zero,and a constant 
temperature is maintained at its ends 
u(0,t)=U1=const ; u(L0,t)=U2=const ; 0 < t < 1: 
Find the temperature u(x; t) of the rod for t > 0.""")   

for ent in doc.ents: 
    print(ent.text,ent.start_char,ent.end_char,ent.label_)
spacy.displacy.serve(doc,style='ent')
\end{lstlisting}
\newpage
The code snippet above shows a document object which incorporates a labelled text being outputted from as seen on \ref{fig:NLP Visulaizationl} from a trained NLP object. Please refer to appendix \ref{Code} to look into the detail implementation of the NLP object. The whole project code can be found in the jupyter notebook file, which is uploaded in the GitHub repository (formulation\_math1.ipynb).

\begin{figure}[hbt!]
    \centering
    \includegraphics[width=1.0\textwidth]{images/displacy_NER_result.png}
    \caption{visualization of NER model output}
    \label{fig:NLP Visulaizationl}
\end{figure} 
From figure \ref{fig:NLP Visulaizationl} above we can see the initial condition label was missed by our model so we have to retrain it until we have the most possible fit. 
Because it might be an overdue/overkill  to use a machine learning algorithm to classify the problems, we used a basic conditional statements to overcome the classification.

For the functional equation formulation the regex can only handles patterns of the problems given and a variation or change of pattern must be added to handle new patterns and conditions. 

\chapter{Evaluation}

Using our method of mixing regex based and deep learning models to formulate mathematical equation from a given mathematical problem, we get a result that is not perfect but promising. 
Since the core of information extraction is used by the help of deep learning, we use the common methods, which is using precision, recall and F1-score to evaluate our model. We also incorporated conditional statements and regex to extract trigger words but we wont be evaluating that since its not part of the deep learning model. 

\begin{description}
\item[Precision]\hfill \\
Precision by definition is the number correctly predicted label out of the whole label prediction. 
 
 \[ precision=\frac{TP}{TP+FP}\]

In our context for instance the precision for the label boundary value would be the number of correctly predicted boundary value over the every predicted boundary value.

\item[Recall]\hfill \\
Recall is the number of true actual label out of all actual label.

\[precision=\frac{TP}{TP+FN}\]

Meaning how many of the actual label was predicted correctly. In comparison of precision, which measures quality recall indicates quantity of the model.

\item[F1-score]\hfill \\
With precision focusing on the prediction proportions and recall focusing on actual labels we have a F1-score measure that balances between the two measures and its given by the following equation:

    \[F1=\frac{2*recall*precision}{precision+recall}\]




\end{description}

\newpage
With our model for the predicted two labels of boundary condition and initial condition we have the following measures. 


\begin{table}[ht]
    \begin{center}
    \label{tab:my_label}
    \begin{tabular}{p{100pt}|p{60pt}|p{60pt}|p{60pt}}
        Label    & Precision &   Recall    &    F1-score \\
        \hline
        boundary condition &66.66\% & 54.54\%& 60.0\% \\
        initial condition &100\% & 50\% & 66.66\%

    \end{tabular}
    \end{center}
    \caption{PRF measure of our model}

\end{table}

Considering the complexity of the problem and the limited data used around 60\% of F1 score, as shown in table \ref{tab:my_label}, is satisfactory. These results are similar to other approaches such as \cite{wang2017deep} et al., which are in the range of 60\%-80\%. Nevertheless the achieved accuracy is not high enough to use this model for a reliable prediction, considering that 1 is perfect and 0 total failure.

\chapter{Conclusion}
Based on current research, we present an approach to formulating mathematical equation from a mathematical problem. We used a combination of regex and deep learning to extract boundary conditions and trigger words that determine type of word problems. With the help of customizable NER model of spaCy which is powered by deep learning algorithm \textbf{\textit{encode, embed, attend, predict}}, we were able to label and use the labelled data as input for reformulating.
\par This NER model can be scaled to determine the given conditions or values or inequalities of problems within a narrow scope, with each new data introduced to the model needs to be trained to be able to learn and predict on a test data. In that regard, our model is merely of a demonstration of methods and does not identify the whole attributes of the given problems due to the invariance between the problems and the limited data which was available.
\par For future work if the text structure is limited to a minimum and also consider the types of problems a more stable problem formulating model can be built with deep learning. 

\printbibliography

\appendix
\chapter{Problem statements}
\label{problems}

In this chapter the focus is on the math problems we consider to solve. The examples of mathematical expression appearing in engineering problem formulations. In general, these expressions can be sub-divided into the following categories:
\begin{enumerate}
\item {\bfseries Variables}: $x$, $t$, $z$, $\epsilon$;
\item {\bfseries Constants}: $v_{0}$, $C_{0}$, $u_{1}=const$, $u_{0}=const$, $U_{0}=const$, $A=const$, $Q$, $x_{0}$, $h$,  $v_{0}=const$, $E_{0}$, $R_{0}$, $b$;
\item {\bfseries Functional expressions}: $\varphi(t)$, $f(x)$, $u(x,t)$, $F(x,t)$;
\item {\bfseries Equations}: $t=0$, $x=l^{''}$, $f(x)=U_{0}=const$, $x=l/2$, $u(0,t)=U_{1}=const$, $u(l,t)=U_{2}=const$, $U_{1}=U_{2}=0$, $f(x)=0$, $x=0$, $x=l$, $u(l,t)=At$, $u(0,t)=0$;
\item {\bfseries Inequalities}: $0\leq x \leq l$, $0\leq x \leq l^{''}$, $\epsilon >0$, $0<t<+ \infty$, $t>0$, $0<x_{0}<l$, $0\leq x\leq x_{0}$, $x_{0}\leq x \leq l$, $0<x_{0}<l$;
\item {\bfseries More general formulae}: $k_{x}$, $k_{t}$, $k_{u}$, $\bar{u}(x)=\lim\limits_{t\to +\infty} u(x,t)$, $\Phi (t)\sin \frac{\pi x}{l}$, $q(t)=Ae^{-ht}$, $u_{t}=a^{2}u_{xx}-hu$, $u_{t}=a^{2}u_{xx}-Hu+f(x,t)$, $u_{x}(0,t)-hu(0,t)=\psi_{1}(t)$, $u_{x}(l,t)+hu(l,t)=\psi_{2}(t)$, $u(x,0)=\varphi (x)$, $t \to +\infty$, $u(l,t)=A \cos \omega t$, $u_{x}(l,t)=A \cos \omega t$, $u_{x}(l,t)+hu(l,t)=A \cos \omega t$, $y=Ae^{-mx}$, $u_{t}=a^{2}u_{xx}+bu+f(x,t)$.
\end{enumerate}
As we see from the above examples, a lot of expressions have the same meaning but written in different form: for example $v_{0}$ and $v_{0}=const$, $t>0$ and $0<t<+\infty$. Therefore, as in the case of any formal language, we need fix a kind of syntax for expressions which are allowed in our model formulations.\par

\section{Examples of problem formulations}

\begin{enumerate}
\item The initial temperature of a rod $0<x<l$ is an arbitrary function of $x$. The temperatures of the ends are constant:
\begin{equation*}
u(0,t) = U_{1} = \mathrm{const}, \quad u(l,t) = U_{2} = \mathrm{const}, \quad 0<t<\infty.
\end{equation*}
A heat exchange obeying Newton's law takes place at the surface with a medium whose temperature equals $u_{0}=\mathrm{const}$. Find the temperature of the rod. Consider, in particular, the case where $U_{1}=U_{2}=0$.
\item Determine the temperature of a rod $0<x<l$ thermally insulated along its surface, if one of its ends ($x=0$) is kept at a given fixed temperature, and a given constant heat flow is maintained at the other end ($x=l$), the initial temperature being arbitrary.
\item Find the temperature of a rod $0<x<l$ thermally insulated along the surface, if heat sources of density equal to $\Phi(t)\sin\left(\dfrac{\pi x}{l}\right)$ are continuously distributed over the rod, and the initial temperature of the rod is an arbitrary function $f(x)$ and the temperature of the ends is maintained equal to zero.
\item Find the distribution of temperature in a rod $0\leq x\leq l$ thermally insulated along the surface, if the temperature of its ends is maintained to zero, and the initial temperature equals an arbitrary function $f(x)$.
\item Find the distribution of temperature in a rod $0\leq x\leq l$ thermally insulated along the surface, if the temperature of its ends is maintained to zero, and the initial temperature equals is constant $U_{0}=const$.
\item The initial temperature of a rod $0\leq x\leq l$ thermally insulated along the surface equals $U_{0}=const$, and a constant temperature is maintained at its ends 
\begin{equation*}
u(0,t)=U_{1}=const, \qquad u(l,t)=U_{2}=const, \qquad 0<t<+\infty.
\end{equation*}
Find the temperature $u(x,t)$ of the rod for $t>0$.
\item Find the distribution of temperature in a rod of length $l$ thermally insulated along the surface, if the temperature of its one end ($x=0$) is maintained at constant value $U_{1}$, and the temperature of another end ($x=l$) is zero, and the initial temperature equals an arbitrary function $\varphi(x)$.
\item Find the distribution of temperature in a rod of length $l$ thermally insulated along the surface, if the temperature of its one end ($x=0$) is zero, and the temperature of another end ($x=l$) is constant $U_{0}$, and the initial temperature is zero.
\item The initial temperature of a rod $0\leq x\leq l$ thermally insulated along the surface equals zero, and a constant temperature is maintained at its ends 
\begin{equation*}
u(0,t)=U_{1}=const, \qquad u(l,t)=U_{2}=const, \qquad 0<t<+\infty.
\end{equation*}
Find the temperature $u(x,t)$ of the rod for $t>0$.
\item Find the temperature distribution in a rod $0<x<l$ thermally insulated along the surface, if a temperature, equal to zero, is maintained at its end $x=0$ and at the end $x=l$ the temperature varies according to the law
\begin{equation*}
u(l,t)=At, \qquad A=const, \qquad 0<t<+\infty.
\end{equation*}
The initial temperature of the rod equals zero.
\end{enumerate}
To simplify the task of text recognition, we set up a basic collection of symbols which are allowed to be used in problem formulations:
\begin{enumerate}
\item {\bfseries Variables}: only $x$, $y$, $z$, $t$ are allowed as variables;
\item {\bfseries Constants}: all constants must be written by upper case letters with a sub-index (0, 1, or 2), e.g. $U_{0}$, $A_{1}$;
\item {\bfseries Functional expressions}: functions names must be written by lower case letters or by Greek letters, and arguments in parentheses must follow the function name, e.g. $\varphi(t)$, $f(x)$, $u(x,t)$, $g(x,y,t)$;
\item {\bfseries Equations}: equations must be written only with one equality sign, i.e. expressions such as $f(x)=U_{0}=const$ are not allowed; equations can be written for variables, e.g. $t=0$ or $x=l$, and for functions, e.g. $f(x)=U_{0}$, $u(0,t)$;
\item {\bfseries Inequalities}: inequalities must be written according to rules 1-4 introduced above, and can be one-sided inequality, i,e. $t>0$ or $x<L_{0}$, or two-sided inequalities, i.e. $0<x<l$, $0<t<\infty$.
\end{enumerate}\par
According to the five rules introduced above, the problem formulations can be now reformulated as follows:
\begin{enumerate}
\item The initial temperature of a rod $0<x<L_{0}$ is an arbitrary function of $x$. The temperatures of the ends are constant:
\begin{equation*}
u(0,t) = U_{1}, \quad u(L_{0},t) = U_{2}, \quad 0<t<\infty.
\end{equation*}
A heat exchange obeying Newton's law takes place at the surface with a medium whose temperature equals $U_{0}$. Find the temperature of the rod. Consider, in particular, the case where $U_{1}=0$ and $U_{2}=0$.
\item Determine the temperature of a rod $0<x<L_{0}$ thermally insulated along its surface, if one of its ends ($x=0$) is kept at a given fixed temperature, and a given constant heat flow is maintained at the other end ($x=L_{0}$), the initial temperature being arbitrary.
\item Find the temperature of a rod $0<x<L_{0}$ thermally insulated along the surface, if heat sources of density equal to $\phi(t)\sin\left(\dfrac{\pi x}{L_{0}}\right)$ are continuously distributed over the rod, and the initial temperature of the rod is an arbitrary function $f(x)$ and the temperature of the ends is maintained equal to zero.
\item Find the distribution of temperature in a rod $0\leq x\leq L_{0}$ thermally insulated along the surface, if the temperature of its ends is maintained to zero, and the initial temperature equals an arbitrary function $f(x)$.
\item Find the distribution of temperature in a rod $0\leq x\leq L_{0}$ thermally insulated along the surface, if the temperature of its ends is maintained to zero, and the initial temperature equals is $U_{0}$.
\item The initial temperature of a rod $0< x< L_{0}$ thermally insulated along the surface equals $U_{0}$, and a constant temperature is maintained at its ends 
\begin{equation*}
u(0,t)=U_{1}, \qquad u(L_{0},t)=U_{2}, \qquad 0<t<\infty.
\end{equation*}
Find the temperature $u(x,t)$ of the rod for $t>0$.
\item Find the distribution of temperature in a rod of length $L_{0}$ thermally insulated along the surface, if the temperature of its one end ($x=0$) is maintained at value $U_{1}$, and the temperature of another end ($x=L_{0}$) is zero, and the initial temperature equals an arbitrary function $\varphi(x)$.
\item Find the distribution of temperature in a rod of length $L_{0}$ thermally insulated along the surface, if the temperature of its one end ($x=0$) is zero, and the temperature of another end ($x=L_{0}$) is constant $U_{0}$, and the initial temperature is zero.
\item The initial temperature of a rod $0< x< L_{0}$ thermally insulated along the surface equals zero, and a constant temperature is maintained at its ends 
\begin{equation*}
u(0,t)=U_{1}, \qquad u(L_{0},t)=U_{2}, \qquad 0<t<\infty.
\end{equation*}
Find the temperature $u(x,t)$ of the rod for $t>0$.
\item Find the temperature distribution in a rod $0<x<L_{0}$ thermally insulated along the surface, if a temperature, equal to zero, is maintained at its end $x=0$ and at the end $x=L_{0}$ the temperature varies according to the law
\begin{equation*}
u(L_{0},t)=A_{0}t, \quad 0<t<\infty.
\end{equation*}
The initial temperature of the rod equals zero.
\end{enumerate}\par

\section{Understanding semantic of problem formulations}

In this section, we will analyse the problem formulations introduced above w.r.t. semantic of words and their connection to mathematical expressions. The goal of the modelling process is to formulate a mathematical problem, which needed to be solved, or in other words, to formulate an initial boundary value problem. For formulating the initial boundary value problem it is necessary to address the following questions:
\begin{enumerate}
\item What type of differential equation is considered (parabolic, hyperbolic, elliptic)?
\item Is the differential equation homogeneous or non-homogeneous?
\item How boundary conditions look like?
\item Is the problem static?
\item In the case of a dynamic problem, how initial conditions look like?
\end{enumerate}

\subsection{Problem 1}

{\bfseries Formulation}: The initial temperature of a rod $0<x<L_{0}$ is an arbitrary function of $x$. The temperatures of the ends are constant:
\begin{equation*}
u(0,t) = U_{1}, \quad u(L_{0},t) = U_{2}, \quad 0<t<\infty.
\end{equation*}
A heat exchange obeying Newton's law takes place at the surface with a medium whose temperature equals $U_{0}$. Find the temperature of the rod. Consider, in particular, the case where $U_{1}=0$ and $U_{2}=0$.\par
Following the five questions formulated above, let us now analyse this problem:
\begin{enumerate}
\item The differential equation of a parabolic type is considered. This is indicated by the keyword {\bfseries temperature}, because heat conduction process are typically modelled by parabolic equations. {\bfseries Important}: the parabolic heat equation will become an elliptic equation if a static process is considered.
\item The differential equation is non-homogeneous and this is indicated by the words: {\bfseries A heat exchange obeying Newton's law takes place at the surface with a medium whose temperature equals $U_{0}$}.
\item The boundary conditions are specified in the sentence: {\bfseries The temperatures of the ends are constant}.
\item The problem is time-dependent and this is indicated by words: {\bfseries The initial temperature}, because only time-dependent problems have initial conditions.
\item The initial conditions are specified in the sentence: {\bfseries The initial temperature of a rod $0<x<L_{0}$ is an arbitrary function of $x$}.
\end{enumerate}
Next step is to map the verbal description and the information we have just analysed into mathematical expressions. We will do it step by step:
\begin{itemize}
\item {\bfseries the initial temperature} $\Longrightarrow$ $u(x,0)$;
\item {\bfseries is an arbitrary function of $x$} $\Longrightarrow$ $f(x)$;
\item Thus, the first sentence {\bfseries The initial temperature of a rod $0<x<L_{0}$ is an arbitrary function of $x$.} is mapped to $u(x,0)=f(x), 0<x<L_{0}$.
\item {\bfseries The temperatures of the ends} $\Longrightarrow$ $u(0,t)$ and $u(L_{0},t)$;
\item {\bfseries A heat exchange obeying Newton's law takes place at the surface with a medium whose temperature equals $U_{0}$} $\Longrightarrow$ $-H_{0}(u(x,t)-U_{0})$;
\item {\bfseries Find the temperature of the rod.} $\Longrightarrow$ means we need to find the function $u(x,t)$.
\end{itemize}
Next, from our analysis, especially from points 1 and 4, we know that we need to consider a parabolic equation of the following form
\begin{equation*}
\frac{\partial u}{\partial t} - a^{2}\frac{\partial^{2} u}{\partial x^{2}} = 0.
\end{equation*}
Additionally, from point 2 we know that the equation is non-homogeneous, and therefore, the latter equation is rewritten now as follows
\begin{equation*}
\frac{\partial u}{\partial t} - a^{2}\frac{\partial^{2} u}{\partial x^{2}} = -H_{0}(u(x,t)-U_{0}).
\end{equation*}
Finally, summarising everything above, we get the following mathematical model:
\begin{itemize}
\item Differential equation: $\dfrac{\partial u}{\partial t} - a^{2}\dfrac{\partial^{2} u}{\partial x^{2}} = -H_{0}(u(x,t)-U_{0})$;
\item Boundary conditions: $u(0,t) = U_{1}, u(L_{0},t) = U_{2}$;
\item Initial condition: $u(x,0)=f(x)$;
\item Inequalities for variables: $0<x<L_{0}$ and $0<t<\infty$.
\end{itemize}
As the second model, we get the special case indicated by words {\bfseries Find the temperature of the rod. Consider, in particular, the case where $U_{1}=U_{2}=0$.}. This model is given then by:
\begin{itemize}
\item Differential equation: $\dfrac{\partial u}{\partial t} - a^{2}\dfrac{\partial^{2} u}{\partial x^{2}} = -H_{0}(u(x,t)-U_{0})$;
\item Boundary conditions: $u(0,t) = 0, u(L_{0},t) = 0$;
\item Initial condition: $u(x,0)=f(x)$;
\item Inequalities for variables: $0<x<L_{0}$ and $0<t<\infty$.
\end{itemize}



\subsection{Problem 2}
{\bfseries Formulation}: Determine the temperature of a rod $0<x<L_{0}$ thermally insulated along its surface, if one of its ends ($x=0$) is kept at a given fixed temperature, and a given constant heat flow is maintained at the other end ($x=L_{0}$), the initial temperature being arbitrary.\par
Following again the five questions, for this problem we get:
\begin{enumerate}
\item The differential equation of a parabolic type is considered. This is indicated by the keyword {\bfseries temperature}.
\item The differential equation is homogeneous, because the problem formulation does not contain the words: {\bfseries heat source}, {\bfseries Newton's law}, and {\bfseries Fourier's law}.
\item The boundary conditions are specified in the sentence: {\bfseries if one of its ends ($x=0$) is kept at a given fixed temperature, and a given constant heat flow is maintained at the other end ($x=L_{0}$)}.
\item The problem is time-dependent and this is indicated by words: {\bfseries The initial temperature}.
\item The initial conditions are specified by words: {\bfseries the initial temperature being arbitrary}.
\end{enumerate}
Next step is to map the verbal description and the information we have just analysed into mathematical expressions:
\begin{itemize}
\item {\bfseries Determine the temperature of a rod} $\Longrightarrow$ means we need to find the function $u(x,t)$;
\item {\bfseries if one of its ends ($x=0$) is kept at a given fixed temperature} $\Longrightarrow$ $u(0,t)=U_{0}$;
\item {\bfseries a given constant heat flow is maintained at the other end ($x=l$)} $\Longrightarrow$ $\lambda \dfrac{\partial u}{\partial x}(L_{0},t)=U_{1}$;
\item {\bfseries the initial temperature being arbitrary} is mapped to $u(x,0)=f(x), 0<x<L_{0}$.
\end{itemize}
Next, from our analysis, especially from points 1, 2, and 4, we know that we need to consider a parabolic equation of the following form
\begin{equation*}
\frac{\partial u}{\partial t} - a^{2}\frac{\partial^{2} u}{\partial x^{2}} = 0.
\end{equation*}
Finally, summarising everything above, we get the following mathematical model:
\begin{itemize}
\item Differential equation: $\dfrac{\partial u}{\partial t} = a^{2}\dfrac{\partial^{2} u}{\partial x^{2}}$;
\item Boundary conditions: $u(0,t)=U_{0}, \lambda \dfrac{\partial u}{\partial x}(L_{0},t)=U_{1}$;
\item Initial condition: $u(x,0)=f(x)$;
\item Inequalities for variables: $0<x<L_{0}$ and $0<t<\infty$.
\end{itemize}

\subsection{Problem 3}

{\bfseries Formulation}: Find the temperature of a rod $0<x<L_{0}$ thermally insulated along the surface, if heat sources of density equal to $\phi(t)\sin\left(\dfrac{\pi x}{L_{0}}\right)$ are continuously distributed over the rod, and the initial temperature of the rod is an arbitrary function $f(x)$ and the temperature of the ends is maintained equal to zero.
\begin{enumerate}
\item The differential equation of a parabolic type is considered. This is indicated by the keyword {\bfseries temperature}.
\item The differential equation is non-homogeneous, because the problem formulation contains the words: {\bfseries heat sources}.
\item The boundary conditions are specified in the sentence: {\bfseries the temperature of the ends is maintained equal to zero}.
\item The problem is time-dependent and this is indicated by words: {\bfseries The initial temperature}.
\item The initial conditions are specified by words: {\bfseries the initial temperature of the rod is an arbitrary function $f(x)$}.
\end{enumerate}
Next step is to map the verbal description and the information we have just analysed into mathematical expressions:
\begin{itemize}
\item {\bfseries Find the temperature of a rod} $\Longrightarrow$ means we need to find the function $u(x,t)$;
\item {\bfseries if heat sources of density equal to $\phi(t)\sin\left(\dfrac{\pi x}{L_{0}}\right)$ are continuously distributed over the rod} $\Longrightarrow$ means that the right-hand side of the equation is $\phi(t)\sin\left(\dfrac{\pi x}{L_{0}}\right)$;
\item {\bfseries the initial temperature of the rod is an arbitrary function $f(x)$} $\Longrightarrow$ $u(x,0)=f(x), 0<x<L_{0}$;
\item {\bfseries the temperature of the ends is maintained equal to zero} $\Longrightarrow$ $u(0,t)=0$ and $u(L_{0},t)=0$.
\end{itemize}
Finally, summarising everything above, we get the following mathematical model:
\begin{itemize}
\item Differential equation: $\dfrac{\partial u}{\partial t} - a^{2}\dfrac{\partial^{2} u}{\partial x^{2}}=\phi(t)\sin\left(\dfrac{\pi x}{L_{0}}\right)$;
\item Boundary conditions: $u(0,t)=0, u(L_{0},t)=0$;
\item Initial condition: $u(x,0)=f(x)$;
\item Inequalities for variables: $0<x<L_{0}$ and $0<t<\infty$.
\end{itemize}

\subsection{Problem 4}

{\bfseries Formulation}: Find the distribution of temperature in a rod $0\leq x\leq L_{0}$ thermally insulated along the surface, if the temperature of its ends is maintained to zero, and the initial temperature equals an arbitrary function $f(x)$.
\begin{enumerate}
\item The differential equation of a parabolic type is considered. This is indicated by the keyword {\bfseries temperature}.
\item The differential equation is homogeneous, because the problem formulation does not contain the words: {\bfseries heat source}, {\bfseries Newton's law}, and {\bfseries Fourier's law}.
\item The boundary conditions are specified in the sentence: {\bfseries the temperature of its ends is maintained to zero}.
\item The problem is time-dependent and this is indicated by words: {\bfseries the initial temperature}.
\item The initial conditions are specified by words: {\bfseries the initial temperature equals an arbitrary function $f(x)$}.
\end{enumerate}
Next step is to map the verbal description and the information we have just analysed into mathematical expressions:
\begin{itemize}
\item {\bfseries Find the distribution of temperature in a rod} $\Longrightarrow$ means we need to find the function $u(x,t)$;
\item {\bfseries the temperature of the ends is maintained to zero} $\Longrightarrow$ $u(0,t)=0$ and $u(L_{0},t)=0$
\item {\bfseries the initial temperature equals an arbitrary function $f(x)$} $\Longrightarrow$ $u(x,0)=f(x), 0<x<L_{0}$.
\end{itemize}
Finally, summarising everything above, we get the following mathematical model:
\begin{itemize}
\item Differential equation: $\dfrac{\partial u}{\partial t} - a^{2}\dfrac{\partial^{2} u}{\partial x^{2}}=0$;
\item Boundary conditions: $u(0,t)=0, u(L_{0},t)=0$;
\item Initial condition: $u(x,0)=f(x)$;
\item Inequalities for variables: $0<x<L_{0}$ and $0<t<\infty$.
\end{itemize}

\subsection{Problem 5}

{\bfseries Formulation}: Find the distribution of temperature in a rod $0\leq x\leq L_{0}$ thermally insulated along the surface, if the temperature of its ends is maintained to zero, and the initial temperature equals is $U_{0}$.
\begin{enumerate}
\item The differential equation of a parabolic type is considered. This is indicated by the keyword {\bfseries temperature}.
\item The differential equation is homogeneous, because the problem formulation does not contain the words: {\bfseries heat source}, {\bfseries Newton's law}, and {\bfseries Fourier's law}.
\item The boundary conditions are specified in the sentence: {\bfseries the temperature of its ends is maintained to zero}.
\item The problem is time-dependent and this is indicated by words: {\bfseries the initial temperature}.
\item The initial conditions are specified by words: {\bfseries the initial temperature equals is $U_{0}$}.
\end{enumerate}
Next step is to map the verbal description and the information we have just analysed into mathematical expressions:
\begin{itemize}
\item {\bfseries Find the distribution of temperature in a rod} $\Longrightarrow$ means we need to find the function $u(x,t)$;
\item {\bfseries the temperature of the ends is maintained to zero} $\Longrightarrow$ $u(0,t)=0$ and $u(L_{0},t)=0$
\item {\bfseries the initial temperature equals is $U_{0}$} $\Longrightarrow$ $u(x,0)=U_{0}, 0<x<L_{0}$.
\end{itemize}
Finally, summarising everything above, we get the following mathematical model:
\begin{itemize}
\item Differential equation: $\dfrac{\partial u}{\partial t} - a^{2}\dfrac{\partial^{2} u}{\partial x^{2}}=0$;
\item Boundary conditions: $u(0,t)=0, u(L_{0},t)=0$;
\item Initial condition: $u(x,0)=U_{0}$;
\item Inequalities for variables: $0<x<L_{0}$ and $0<t<\infty$.
\end{itemize}

\subsection{Problem 6}

{\bfseries Formulation}: The initial temperature of a rod $0< x< L_{0}$ thermally insulated along the surface equals $U_{0}$, and a constant temperature is maintained at its ends 
\begin{equation*}
u(0,t)=U_{1}, \qquad u(L_{0},t)=U_{2}, \qquad 0<t<\infty.
\end{equation*}
Find the temperature $u(x,t)$ of the rod for $t>0$.
\begin{enumerate}
\item The differential equation of a parabolic type is considered. This is indicated by the keyword {\bfseries temperature}.
\item The differential equation is homogeneous, because the problem formulation does not contain the words: {\bfseries heat source}, {\bfseries Newton's law}, and {\bfseries Fourier's law}.
\item The boundary conditions are specified in the sentence: {\bfseries a constant temperature is maintained at its ends}.
\item The problem is time-dependent and this is indicated by words: {\bfseries the initial temperature}.
\item The initial conditions are specified by words: {\bfseries The initial temperature of a rod $0< x< L_{0}$ thermally insulated along the surface equals $U_{0}$}.
\end{enumerate}
Next step is to map the verbal description and the information we have just analysed into mathematical expressions:
\begin{itemize}
\item {\bfseries The initial temperature of a rod $0< x< L_{0}$ thermally insulated along the surface equals $U_{0}$} $\Longrightarrow$ $u(x,0)=U_{0}, 0<x<L_{0}$;
\item {\bfseries  a constant temperature is maintained at its ends} $\Longrightarrow$ $u(0,t)=U_{1}$ and $u(L_{0},t)=U_{2}$;
\item {\bfseries Find the temperature} $\Longrightarrow$ means we need to find the function $u(x,t)$.
\end{itemize}
Finally, summarising everything above, we get the following mathematical model:
\begin{itemize}
\item Differential equation: $\dfrac{\partial u}{\partial t} - a^{2}\dfrac{\partial^{2} u}{\partial x^{2}}=0$;
\item Boundary conditions: $u(0,t)=U_{1}$, $u(L_{0},t)=U_{2}$;
\item Initial condition: $u(x,0)=U_{0}$;
\item Inequalities for variables: $0<x<L_{0}$ and $0<t<\infty$.
\end{itemize}

\subsection{Problem 7}

{\bfseries Formulation}: Find the distribution of temperature in a rod of length $L_{0}$ thermally insulated along the surface, if the temperature of its one end ($x=0$) is maintained at value $U_{1}$, and the temperature of another end ($x=L_{0}$) is zero, and the initial temperature equals an arbitrary function $\varphi(x)$.
\begin{enumerate}
\item The differential equation of a parabolic type is considered. This is indicated by the keyword {\bfseries temperature}.
\item The differential equation is homogeneous, because the problem formulation does not contain the words: {\bfseries heat source}, {\bfseries Newton's law}, and {\bfseries Fourier's law}.
\item The boundary conditions are specified in the sentence: {\bfseries if the temperature of its one end ($x=0$) is maintained at value $U_{1}$, and the temperature of another end ($x=L_{0}$) is zero}.
\item The problem is time-dependent and this is indicated by words: {\bfseries the initial temperature}.
\item The initial conditions are specified by words: {\bfseries the initial temperature equals an arbitrary function $\varphi(x)$}.
\end{enumerate}
Next step is to map the verbal description and the information we have just analysed into mathematical expressions:
\begin{itemize}
\item {\bfseries Find the distribution of temperature in a rod of length $L_{0}$ thermally insulated along the surface} $\Longrightarrow$ means we need to find the function $u(x,t)$;
\item {\bfseries the temperature of its one end ($x=0$) is maintained at value $U_{1}$, and the temperature of another end ($x=L_{0}$) is zero} $\Longrightarrow$ $u(0,t)=U_{1}$ and $u(L_{0},t)=0$
\item {\bfseries the initial temperature equals an arbitrary function $\varphi(x)$} $\Longrightarrow$ $u(x,0)=\varphi(x), 0<x<L_{0}$.
\end{itemize}
Finally, summarising everything above, we get the following mathematical model:
\begin{itemize}
\item Differential equation: $\dfrac{\partial u}{\partial t} - a^{2}\dfrac{\partial^{2} u}{\partial x^{2}}=0$;
\item Boundary conditions: $u(0,t)=U_{1}, u(L_{0},t)=0$;
\item Initial condition: $u(x,0)=\varphi(x)$;
\item Inequalities for variables: $0<x<L_{0}$ and $0<t<\infty$.
\end{itemize}

\subsection{Problem 8}

{\bfseries Formulation}: Find the distribution of temperature in a rod of length $L_{0}$ thermally insulated along the surface, if the temperature of its one end ($x=0$) is zero, and the temperature of another end ($x=L_{0}$) is constant $U_{0}$, and the initial temperature is zero.
\begin{enumerate}
\item The differential equation of a parabolic type is considered. This is indicated by the keyword {\bfseries temperature}.
\item The differential equation is homogeneous, because the problem formulation does not contain the words: {\bfseries heat source}, {\bfseries Newton's law}, and {\bfseries Fourier's law}.
\item The boundary conditions are specified in the sentence: {\bfseries if the temperature of its one end ($x=0$) is zero, and the temperature of another end ($x=L_{0}$) is constant $U_{0}$}.
\item The problem is time-dependent and this is indicated by words: {\bfseries the initial temperature}.
\item The initial conditions are specified by words: {\bfseries the initial temperature is zero}.
\end{enumerate}
Next step is to map the verbal description and the information we have just analysed into mathematical expressions:
\begin{itemize}
\item {\bfseries Find the distribution of temperature in a rod of length $L_{0}$ thermally insulated along the surface} $\Longrightarrow$ means we need to find the function $u(x,t)$;
\item {\bfseries if the temperature of its one end ($x=0$) is zero, and the temperature of another end ($x=L_{0}$) is constant $U_{0}$} $\Longrightarrow$ $u(0,t)=0$ and $u(L_{0},t)=U_{0}$
\item {\bfseries the initial temperature is zero} $\Longrightarrow$ $u(x,0)=0, 0<x<L_{0}$.
\end{itemize}
Finally, summarising everything above, we get the following mathematical model:
\begin{itemize}
\item Differential equation: $\dfrac{\partial u}{\partial t} - a^{2}\dfrac{\partial^{2} u}{\partial x^{2}}=0$;
\item Boundary conditions: $u(0,t)=0, u(L_{0},t)=U_{0}$;
\item Initial condition: $u(x,0)=0$;
\item Inequalities for variables: $0<x<L_{0}$ and $0<t<\infty$.
\end{itemize}

\subsection{Problem 9}

{\bfseries Formulation}: The initial temperature of a rod $0< x< L_{0}$ thermally insulated along the surface equals zero, and a constant temperature is maintained at its ends 
\begin{equation*}
u(0,t)=U_{1}, \qquad u(L_{0},t)=U_{2}, \qquad 0<t<\infty.
\end{equation*}
Find the temperature $u(x,t)$ of the rod for $t>0$.
\begin{enumerate}
\item The differential equation of a parabolic type is considered. This is indicated by the keyword {\bfseries temperature}.
\item The differential equation is homogeneous, because the problem formulation does not contain the words: {\bfseries heat source}, {\bfseries Newton's law}, and {\bfseries Fourier's law}.
\item The boundary conditions are specified in the sentence: {\bfseries a constant temperature is maintained at its ends }.
\item The problem is time-dependent and this is indicated by words: {\bfseries the initial temperature}.
\item The initial conditions are specified by words: {\bfseries The initial temperature of a rod $0< x< L_{0}$ thermally insulated along the surface equals zero}.
\end{enumerate}
Next step is to map the verbal description and the information we have just analysed into mathematical expressions:
\begin{itemize}
\item {\bfseries The initial temperature of a rod $0< x< L_{0}$ thermally insulated along the surface equals zero} $\Longrightarrow$ $u(x,0)=0, 0<x<L_{0}$;
\item {\bfseries  a constant temperature is maintained at its ends} $\Longrightarrow$ $u(0,t)=U_{1}$ and $u(L_{0},t)=U_{2}$;
\item {\bfseries Find the temperature} $\Longrightarrow$ means we need to find the function $u(x,t)$.
\end{itemize}
Finally, summarising everything above, we get the following mathematical model:
\begin{itemize}
\item Differential equation: $\dfrac{\partial u}{\partial t} - a^{2}\dfrac{\partial^{2} u}{\partial x^{2}}=0$;
\item Boundary conditions: $u(0,t)=U_{1}$, $u(L_{0},t)=U_{2}$;
\item Initial condition: $u(x,0)=0$;
\item Inequalities for variables: $0<x<L_{0}$ and $0<t<\infty$.
\end{itemize}

\subsection{Problem 10}

{\bfseries Formulation}: Find the temperature distribution in a rod $0<x<L_{0}$ thermally insulated along the surface, if a temperature, equal to zero, is maintained at its end $x=0$ and at the end $x=L_{0}$ the temperature varies according to the law
\begin{equation*}
u(L_{0},t)=A_{0}t, \quad 0<t<\infty.
\end{equation*}
The initial temperature of the rod equals zero.
\begin{enumerate}
\item The differential equation of a parabolic type is considered. This is indicated by the keyword {\bfseries temperature}.
\item The differential equation is homogeneous, because the problem formulation does not contain the words: {\bfseries heat source}, {\bfseries Newton's law}, and {\bfseries Fourier's law}.
\item The boundary conditions are specified in the sentence: {\bfseries if a temperature, equal to zero, is maintained at its end $x=0$ and at the end $x=L_{0}$ the temperature varies according to the law}.
\item The problem is time-dependent and this is indicated by words: {\bfseries the initial temperature}.
\item The initial conditions are specified by words: {\bfseries The initial temperature of the rod equals zero.}.
\end{enumerate}
Next step is to map the verbal description and the information we have just analysed into mathematical expressions:
\begin{itemize}
\item {\bfseries Find the temperature distribution in a rod $0<x<L_{0}$ thermally insulated along the surface} $\Longrightarrow$ means we need to find the function $u(x,t)$;
\item {\bfseries  if a temperature, equal to zero, is maintained at its end $x=0$} $\Longrightarrow$ $u(0,t)=0$;
\item {\bfseries  and at the end $x=L_{0}$ the temperature varies according to the law $u(L_{0},t)=A_{0}t$} $\Longrightarrow$ $u(L_{0},t)=A_{0}t$;
\item {\bfseries The initial temperature of the rod equals zero.} $\Longrightarrow$ $u(x,0)=0, 0<x<L_{0}$;
\end{itemize}
Finally, summarising everything above, we get the following mathematical model:
\begin{itemize}
\item Differential equation: $\dfrac{\partial u}{\partial t} - a^{2}\dfrac{\partial^{2} u}{\partial x^{2}}=0$;
\item Boundary conditions: $u(0,t)=0$, $u(L_{0},t)=A_{0}t$;
\item Initial condition: $u(x,0)=0$;
\item Inequalities for variables: $0<x<L_{0}$ and $0<t<\infty$.
\end{itemize}

\chapter{Code}
\label{Code}
\definecolor{codegreen}{rgb}{0,0.6,0}
\definecolor{codegray}{rgb}{0.5,0.5,0.5}
\definecolor{codepurple}{rgb}{0.58,0,0.82}
\definecolor{backcolour}{rgb}{0.95,0.95,0.92}

\lstdefinestyle{mystyle}{
    backgroundcolor=\color{backcolour},   
    commentstyle=\color{codegreen},
    keywordstyle=\color{magenta},
    numberstyle=\tiny\color{codegray},
    stringstyle=\color{codepurple},
    basicstyle=\ttfamily\footnotesize,
    breakatwhitespace=false,         
    breaklines=true,                 
    captionpos=b,                    
    keepspaces=true,                 
    numbers=left,                    
    numbersep=5pt,                  
    showspaces=false,                
    showstringspaces=false,
    showtabs=false,                  
    tabsize=2
}
\lstset{style=mystyle}
\begin{lstlisting}[language=python]
import spacy 
import random 
from spacy.util import minibatch,compounding 
from pathlib import Path 
from spacy import displacy 
import re 
from spacy.gold import GoldParse
from spacy.scorer import Scorer
from pathlib import Path

nlp=spacy.load('en_core_web_sm')
ner=nlp.get_pipe("ner")
for _,annotations in train:
    for ent in annotations.get("entities"):
        ner.add_label(ent[2])
        
disable_pipes=[pipe for pipe in nlp.pipe_names if pipe !='ner']
#training the model#
def training(train,interation,nlp):
    
    with nlp.disable_pipes(*disable_pipes):
        optimizer=nlp.begin_training()
        for iteration in range(100):
            random.shuffle(train)
            losses={}
            batches=minibatch(train,size=compounding(1.0 ,4.0 ,1.001))
            for batch in batches:
                texts,annotations=zip(*batch)
                nlp.update(
                            texts,
                            annotations,
                            drop=0.2,
                            losses=losses,
                            sgd=optimizer)
                print("losses",losses)
#reformulates input text#
def reformulate(text):
    
    pattern1=re.compile('F\([a-z]\)=.*?(?=,)')
    pattern2=re.compile('F\([a-z],t\)=F\([a-z]\).*?(?= )')

    match=[re.search(pattern,text).group() for pattern in [pattern1,pattern2] if re.search(pattern,text) is not None]
    #match[0]
    if 'temprature' in text:      #the problem is of parabolic type 
        DFE='du/dt-a*a*d(du/dt)'
        if 'heat source' or "Newton's law" or "Fourier's law"  not in text: #equation is homogenous
            print("diffential equation: "+'du/dt=a*a*d(du/dt)/dt')
        else:
            DFE='non-homogenous' 
        
    elif 'vibration' in text:      #problem type of hyperbolic type
        DF='hyperbolic'
        if 'force' or 'excitaion' not in text:    #problem is homogenous
            print("diffential equation: "+'du/dt=a*a*d(du/dt)/dt')
        else:
            print("diffential equation: "+'du/dt=a*a*d(du/dt)/dt'+match[0])
          
    doc=nlp(text)
    for ent in doc.ents:
        #print(ent.text,ent.label_)
        if ent.label_=='boundary_condition':
            print("boundary condition: "+ent.text)

        if ent.label_=='initial_condition':    
            print("initial condition: "+ent.text)
        else:
            print("no initial condition found")
#model evaluation#
def evalu(examples,ner_model):
    scorer=Scorer()
    for input_,annot in examples:
        doc_gold_text=ner_model.make_doc(input_)
        gold=GoldParse(doc_gold_text,entities=annot['entities'])
        pred_value=ner_model(input_)
        scorer.score(pred_value,gold)
    return scorer.scores
print(evalu(test,nlp2))
\end{lstlisting}

\end{document}
